\usepackage[utf8]{inputenc}
\usepackage[T1,T2A]{fontenc}
\usepackage[russian]{babel}
%% Отступы
\usepackage[left=2cm,right=2cm,top=2cm,bottom=2cm,bindingoffset=0cm]{geometry}
\usepackage{enumitem}

%% Картинки
\usepackage{graphicx} % для картинок
\graphicspath{{pic/}}
\usepackage{float} 

%% Точки нумерации заголовков
\usepackage{titlesec}
\titlelabel{\thetitle.\quad}
\usepackage[dotinlabels]{titletoc}

%% Нумерация картинок по секциям
\usepackage{chngcntr}
\counterwithin{figure}{section}
\counterwithin{table}{section}

%% Ставим шрифт Times New Roman
\usepackage{fontspec}
\setmainfont{Times New Roman}

%% Формат картинок
\usepackage{subcaption}
\DeclareCaptionLabelSeparator{custom}{ - }
\captionsetup {labelsep=custom}
\addto\captionsrussian{\renewcommand{\figurename}{Рис.}}
\usepackage[belowskip=2pt,aboveskip=2pt]{caption}
\setlength{\intextsep}{2pt plus 1pt minus 1pt}

%% Изменение отступов
\renewcommand{\baselinestretch}{1}
\setlength{\parskip}{0pt}
\setlength{\parindent}{1.15cm} % Отступ красной строки
\usepackage{titlesec}

% Настройка отступов для всех уровней заголовков
\titlespacing*{\section}{0pt}{0.5cm}{0.25cm}
\titlespacing*{\subsection}{0pt}{0.5cm}{0.25cm}
\titlespacing*{\subsubsection}{0pt}{0.5cm}{0.25cm}

%% Настройка шрифта заголовков
\titleformat{\section}{\normalfont\fontsize{16}{16}\selectfont\bfseries}{\thesection.}{1em}{}
\titleformat{\subsection}{\normalfont\fontsize{14}{14}\selectfont\bfseries}{\thesubsection.}{1em}{}
\titleformat{\subsubsection}{\normalfont\fontsize{13}{12}\selectfont\bfseries}{\thesubsubsection.}{1em}{}
\titleformat{\paragraph}{\normalfont\fontsize{13}{12}\selectfont\bfseries}{\theparagraph.}{1em}{}

%% Изменение оглавления
\usepackage{titletoc}
\titlecontents{section}[0.0cm]{}
{\thecontentslabel\enspace}{}
{{\titlerule*[1pc]{.}\contentspage}}
\titlecontents{subsection}[.4cm]{}
{\thecontentslabel\enspace}{}
{{\titlerule*[1pc]{.}\contentspage}}

\usepackage[unicode, colorlinks=true, linkcolor=black, citecolor=black, filecolor=black, urlcolor=black]{hyperref}

%% Выравнивание без переносов
\usepackage[none]{hyphenat}
\sloppy

%% Настройка нумерации
\usepackage{fancyhdr}
\pagestyle{fancy}
\fancyhf{}
\fancyfoot[R]{\thepage}
\renewcommand{\headrulewidth}{0pt}
\renewcommand{\footrulewidth}{0pt}
\pagenumbering{arabic}

%% Листинги
\usepackage{listings}
\usepackage{xcolor}

\definecolor{codegreen}{rgb}{0,0.6,0}
\definecolor{codegray}{rgb}{0.5,0.5,0.5}
\definecolor{codeblue}{rgb}{0.58,0,0.82}
\definecolor{backcolour}{rgb}{0.95,0.95,0.92}
\lstset{
    backgroundcolor=\color{backcolour},   
    commentstyle=\color{codegreen},
    keywordstyle=\color{blue},
    numberstyle=\tiny\color{codegray},
    stringstyle=\color{codeblue},
    basicstyle=\ttfamily\footnotesize,
    breakatwhitespace=false,         
    breaklines=true,                 
    captionpos=b,                    
    keepspaces=true,                 
    numbers=left,                    
    numbersep=5pt,                  
    showspaces=false,                
    showstringspaces=false,
    showtabs=false,                  
    tabsize=2
}
